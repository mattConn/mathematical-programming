\title{Calculus 3, Exam 2 Formulas}



\documentclass[12pt]{article}

\usepackage[top=.5in, bottom=.75in, left=1in, right=1in]{geometry}
\usepackage{amssymb}
\usepackage{amsmath}
\usepackage{graphicx}
\usepackage{subcaption}


\begin{document}
\maketitle

\section*{Gradients, Directional Derivatives}
\underline{Chain Rule (Alternate Method and Multivariate Method):}\\\\
Given $y= f(x)$, $x= g(t)$, then\\
$$
y = f(x) = f( \ g(t) \ )
$$
and
$$
\frac{dy}{dt} = \frac{dy}{dx} \frac{dx}{dt}
$$
The following is then true:\\\\
Given $z = f(x,y)$, where $x= g(t)$, $y= h(t)$,
$$
z = f(x,y) = f( \ g(t), h(t) \ )
$$

and

$$
\frac{dz}{dt} = \frac{\partial{f}}{\partial{x}} \frac{dx}{dt} \ + \ \frac{\partial{f}}{\partial{y}} \frac{dy}{dt}
$$

\noindent
\underline{Gradient:}\\\\
Given $f(x,y)$, the gradient of $f$ is\\
$$
\nabla f = \left(\frac{\partial{f}}{\partial{x}},\frac{\partial{f}}{\partial{y}}\right)
$$

\noindent
\underline{Directional Derivative:}\\\\
In two variables for $f(x,y)$ in the direction of unit vector $\vec{u}=<a,b>$:\\
$$
{D_{\vec u}}f\left( {x,y} \right) = {f_x}\left( {x,y} \right)a + {f_y}\left( {x,y} \right)b
$$

In three variables for $f(x,y,z)$ in the direction of unit vector $\vec{u}=<a,b,c>$:\\
$$
{D_{\vec u}}f\left( {x,y,z} \right) = {f_x}\left( {x,y,z} \right)a + {f_y}\left( {x,y,z} \right)b + {f_z}\left( {x,y,z} \right)c
$$
\\
\emph{or} in a simpler form:\\
Given $f(x,y)$, the directional derivative of $f$ at $p = (a,b)$ in the direction of unit vector $\vec{u}$ is

$$
{D_{\vec u}}f\left( {p} \right) = {D_{\vec u}}f\left( {a,b} \right) = \nabla f(a,b) \cdot \vec{u}
$$

Rate of change of a function $f$ in the direction of $\nabla f(p)$:\\
$$
||\nabla f(p)||
$$

Rate of change of a function $f$ in the direction of a unit vector $\vec{u}$ making an angle $\theta$ with $\nabla f(p)$:\\
$$
\nabla f(p) \cdot \vec{u} = ||\nabla f(p)|| \ ||\vec{u}|| \ \cos \theta
$$
(This comes from the following identity):
$$
\vec{u} \cdot \vec{v} = ||u|| \ ||v|| \ \cos \theta
$$

\section*{Optimization}
\underline{Critical points}\\
A point $p = (a,b)$ in the domain of $f$ is a \emph{critical point} if:\\\\
$f_x(a,b) = 0$ or $f_x(a,b)$ does not exist, and\\
$f_y(a,b) = 0$ or $f_y(a,b)$ does not exist.\\

Solving the system of $f_x = 0$, $f_y = 0$ will find the critical point (if it exists).\\\\

\underline{Second Derivative Test}\\
The second derivative test finds local max., min., and saddle points.\\
A critical point of $f(x,y)$ is needed, as is the discriminant of $f(x,y)$ which is:
$$
D(a,b) = \begin{bmatrix}
 f_{xx}(a,b) & f_{xy}(a,b)\\
  f_{yx}(a,b) & f_{yy}(a,b)\\
\end{bmatrix}
$$
which yields
$$
D(a,b) = f_{xx}(a,b) f_{yy}(a,b) - f_{xy}^2(a,b)
$$
Then, the second derivative test's rules are:\\\\
 $p = (a,b)$ is a \emph{critical point} of $f(x,y)$.\\
1. If $D > 0$ and $f_{xx}(a,b) > 0$, then $f(a,b)$ is a local minimum.\\
2. If $D > 0$ and $f_{xx}(a,b) < 0$, then $f(a,b)$ is a local maximum.\\
3. If $D < 0$, then $f$ has a saddle point at $(a,b)$.\\
4. If $D = 0$, then the test is inconclusive.\\

\underline{Global Extrema}\\
Let $f(x,y)$ be defined over a closed domain $D$. Then,\\\\
$f$'s extreme values occur at either critical points in the interior of $D$, or\\
at points on the boundary of $D$.\\\\
First, find and examine critical points. Then, evaluate $f$ at the boundaries of $D$.\\
Compare these points to find $f_{max}$ and $f_{min}$.\\

\underline{Lagrange Multipliers}\\
Lagrange multipliers are used to find the min. and max. of a function under a given constraint. Typically, it will look like:\\\\
Given function $f(x,y)$, find the min. and max. under the constraint $g(x,y)$.\\
\indent
($g$ might make a line or other geometric shape; this shape slices $f$. We then need to find the max./min. of $f$ in this slice.)\\\\

The Lagrange condition is:
$$
\nabla f = \lambda \nabla g
$$
Then, solve for $\lambda$ in terms of $x$ and $y$. This might look like:
$$
\lambda = m \ x, \ \lambda = n \ y
$$
(Taking $m$ and $n$ to represent some expressions).\\\\

Then, let these two $\lambda$ expressions equal each other and solve for $x$ and $y$:\\
$$
m \ x = n \ y
$$
This new $x$ and $y$ is the crit. point.\\
Sub. the newly found $x$ and $y$ into constraint $g$ to find max. and min. of $f$.

\section*{Double Integrals}


\end{document}
This is never printed